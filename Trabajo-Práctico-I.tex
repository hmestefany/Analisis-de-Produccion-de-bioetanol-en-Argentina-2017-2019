% Options for packages loaded elsewhere
\PassOptionsToPackage{unicode}{hyperref}
\PassOptionsToPackage{hyphens}{url}
%
\documentclass[
]{article}
\usepackage{amsmath,amssymb}
\usepackage{iftex}
\ifPDFTeX
  \usepackage[T1]{fontenc}
  \usepackage[utf8]{inputenc}
  \usepackage{textcomp} % provide euro and other symbols
\else % if luatex or xetex
  \usepackage{unicode-math} % this also loads fontspec
  \defaultfontfeatures{Scale=MatchLowercase}
  \defaultfontfeatures[\rmfamily]{Ligatures=TeX,Scale=1}
\fi
\usepackage{lmodern}
\ifPDFTeX\else
  % xetex/luatex font selection
\fi
% Use upquote if available, for straight quotes in verbatim environments
\IfFileExists{upquote.sty}{\usepackage{upquote}}{}
\IfFileExists{microtype.sty}{% use microtype if available
  \usepackage[]{microtype}
  \UseMicrotypeSet[protrusion]{basicmath} % disable protrusion for tt fonts
}{}
\makeatletter
\@ifundefined{KOMAClassName}{% if non-KOMA class
  \IfFileExists{parskip.sty}{%
    \usepackage{parskip}
  }{% else
    \setlength{\parindent}{0pt}
    \setlength{\parskip}{6pt plus 2pt minus 1pt}}
}{% if KOMA class
  \KOMAoptions{parskip=half}}
\makeatother
\usepackage{xcolor}
\usepackage[margin=1in]{geometry}
\usepackage{color}
\usepackage{fancyvrb}
\newcommand{\VerbBar}{|}
\newcommand{\VERB}{\Verb[commandchars=\\\{\}]}
\DefineVerbatimEnvironment{Highlighting}{Verbatim}{commandchars=\\\{\}}
% Add ',fontsize=\small' for more characters per line
\usepackage{framed}
\definecolor{shadecolor}{RGB}{248,248,248}
\newenvironment{Shaded}{\begin{snugshade}}{\end{snugshade}}
\newcommand{\AlertTok}[1]{\textcolor[rgb]{0.94,0.16,0.16}{#1}}
\newcommand{\AnnotationTok}[1]{\textcolor[rgb]{0.56,0.35,0.01}{\textbf{\textit{#1}}}}
\newcommand{\AttributeTok}[1]{\textcolor[rgb]{0.13,0.29,0.53}{#1}}
\newcommand{\BaseNTok}[1]{\textcolor[rgb]{0.00,0.00,0.81}{#1}}
\newcommand{\BuiltInTok}[1]{#1}
\newcommand{\CharTok}[1]{\textcolor[rgb]{0.31,0.60,0.02}{#1}}
\newcommand{\CommentTok}[1]{\textcolor[rgb]{0.56,0.35,0.01}{\textit{#1}}}
\newcommand{\CommentVarTok}[1]{\textcolor[rgb]{0.56,0.35,0.01}{\textbf{\textit{#1}}}}
\newcommand{\ConstantTok}[1]{\textcolor[rgb]{0.56,0.35,0.01}{#1}}
\newcommand{\ControlFlowTok}[1]{\textcolor[rgb]{0.13,0.29,0.53}{\textbf{#1}}}
\newcommand{\DataTypeTok}[1]{\textcolor[rgb]{0.13,0.29,0.53}{#1}}
\newcommand{\DecValTok}[1]{\textcolor[rgb]{0.00,0.00,0.81}{#1}}
\newcommand{\DocumentationTok}[1]{\textcolor[rgb]{0.56,0.35,0.01}{\textbf{\textit{#1}}}}
\newcommand{\ErrorTok}[1]{\textcolor[rgb]{0.64,0.00,0.00}{\textbf{#1}}}
\newcommand{\ExtensionTok}[1]{#1}
\newcommand{\FloatTok}[1]{\textcolor[rgb]{0.00,0.00,0.81}{#1}}
\newcommand{\FunctionTok}[1]{\textcolor[rgb]{0.13,0.29,0.53}{\textbf{#1}}}
\newcommand{\ImportTok}[1]{#1}
\newcommand{\InformationTok}[1]{\textcolor[rgb]{0.56,0.35,0.01}{\textbf{\textit{#1}}}}
\newcommand{\KeywordTok}[1]{\textcolor[rgb]{0.13,0.29,0.53}{\textbf{#1}}}
\newcommand{\NormalTok}[1]{#1}
\newcommand{\OperatorTok}[1]{\textcolor[rgb]{0.81,0.36,0.00}{\textbf{#1}}}
\newcommand{\OtherTok}[1]{\textcolor[rgb]{0.56,0.35,0.01}{#1}}
\newcommand{\PreprocessorTok}[1]{\textcolor[rgb]{0.56,0.35,0.01}{\textit{#1}}}
\newcommand{\RegionMarkerTok}[1]{#1}
\newcommand{\SpecialCharTok}[1]{\textcolor[rgb]{0.81,0.36,0.00}{\textbf{#1}}}
\newcommand{\SpecialStringTok}[1]{\textcolor[rgb]{0.31,0.60,0.02}{#1}}
\newcommand{\StringTok}[1]{\textcolor[rgb]{0.31,0.60,0.02}{#1}}
\newcommand{\VariableTok}[1]{\textcolor[rgb]{0.00,0.00,0.00}{#1}}
\newcommand{\VerbatimStringTok}[1]{\textcolor[rgb]{0.31,0.60,0.02}{#1}}
\newcommand{\WarningTok}[1]{\textcolor[rgb]{0.56,0.35,0.01}{\textbf{\textit{#1}}}}
\usepackage{graphicx}
\makeatletter
\def\maxwidth{\ifdim\Gin@nat@width>\linewidth\linewidth\else\Gin@nat@width\fi}
\def\maxheight{\ifdim\Gin@nat@height>\textheight\textheight\else\Gin@nat@height\fi}
\makeatother
% Scale images if necessary, so that they will not overflow the page
% margins by default, and it is still possible to overwrite the defaults
% using explicit options in \includegraphics[width, height, ...]{}
\setkeys{Gin}{width=\maxwidth,height=\maxheight,keepaspectratio}
% Set default figure placement to htbp
\makeatletter
\def\fps@figure{htbp}
\makeatother
\setlength{\emergencystretch}{3em} % prevent overfull lines
\providecommand{\tightlist}{%
  \setlength{\itemsep}{0pt}\setlength{\parskip}{0pt}}
\setcounter{secnumdepth}{-\maxdimen} % remove section numbering
\ifLuaTeX
  \usepackage{selnolig}  % disable illegal ligatures
\fi
\usepackage{bookmark}
\IfFileExists{xurl.sty}{\usepackage{xurl}}{} % add URL line breaks if available
\urlstyle{same}
\hypersetup{
  pdftitle={Análisis de Producción de bioetanol a base de maíz y caña de azucar en Argentina},
  pdfauthor={Estefany Herrera Martinez},
  hidelinks,
  pdfcreator={LaTeX via pandoc}}

\title{Análisis de Producción de bioetanol a base de maíz y caña de
azucar en Argentina}
\author{Estefany Herrera Martinez}
\date{8 de Sep 2023}

\begin{document}
\maketitle

{
\setcounter{tocdepth}{2}
\tableofcontents
}
\subsection{Descripción.}\label{descripciuxf3n.}

Se eligió un dataset que contiene información del volumen producido de
bioetanol (en metros cúbicos) a base de maíz y caña de azucar para la
producción de biocombustible, registrados en Argentina durante los años
2017, 2018 y 2019;cabe resaltar que para el año 2019 tiene información
de los primeros 5 meses. Se descargó el dataset de la página de datos
públicos \url{https://datos.gob.ar} (precisamente, en
\href{https://datos.gob.ar/dataset/agroindustria-biocombustible---produccion-por-insumo-bioetanol}{este
link}) el 8 de Septiembre de 2022. El archivo descargado
(\texttt{Trabajo\ Práctico}) contiene el dataset
\texttt{produccion-de-bioetanol-por-insumo-.csv} que ahora se encuentra
en la carpeta \texttt{datos} y los gráficos de este trabajo estan
guardados en la carpeta output.

El objetivo de este trabajo, es realizar dos visualizaciones del dataset
elegido y visualizar cuál insumo produce mayor cantidad de bioetanol.

\subsection{Importar los datos.}\label{importar-los-datos.}

se importa el dataset de la carpeta datos y se toman las variables
relevantes para el análisis : año, mes, producción de bioetanol. Se usa
del paquete \emph{introdataviz} la función \emph{stack} que genera una
variable de producción de bioetanol y otra variable que clasifica por el
tipo de insumo, a esta función se le agrega las variables de los meses y
años con la función \emph{cbind} obteniendo un dataset que tiene 4
variables ( mes, año, producción y tipo de insumo).Para que el dataset
quede prolijo,se les cambia los nombres a las variables por lo que
corresponde: ``Mes'', ``Año'', ``Producción'' e ``Insumo''.

\begin{verbatim}
##    Año     Mes Produccion Insumo
## 1 2017   Enero      37794   caña
## 2 2017 Febrero      35607   caña
## 3 2017   Marzo      38696   caña
## 4 2017   Abril      35355   caña
## 5 2017    Mayo      44053   caña
## 6 2017   Junio      51159   caña
\end{verbatim}

\subsection{Producción de bioetanol por
insumo.}\label{producciuxf3n-de-bioetanol-por-insumo.}

Se desea observar por medio de gráficos la producción de bioetanol según
el tipo de materia prima , por lo tanto, se utiliza la libreria de
ggplot2 para su elaboración.

\subsubsection{Código}\label{cuxf3digo}

\begin{Shaded}
\begin{Highlighting}[]
\NormalTok{altura}\OtherTok{\textless{}{-}}\NormalTok{ .}\DecValTok{1}

\NormalTok{ g1}\OtherTok{\textless{}{-}}\FunctionTok{ggplot}\NormalTok{(Bioetanol,}
       \FunctionTok{aes}\NormalTok{(}\AttributeTok{x =}\StringTok{""}\NormalTok{ ,}
           \AttributeTok{y =}\NormalTok{ Produccion,}
           \AttributeTok{fill =}\NormalTok{ Insumo)) }\SpecialCharTok{+}
  \CommentTok{\# para dibujar la distirbucion }
\NormalTok{  introdataviz}\SpecialCharTok{::}\FunctionTok{geom\_flat\_violin}\NormalTok{(}\AttributeTok{trim=}\ConstantTok{FALSE}\NormalTok{,}
                                 \AttributeTok{alpha =} \FloatTok{0.4}\NormalTok{,}
                                 \AttributeTok{position =} \FunctionTok{position\_nudge}\NormalTok{(}\AttributeTok{x =}\NormalTok{ altura}\FloatTok{+.05}\NormalTok{)) }\SpecialCharTok{+}
  \CommentTok{\# dibujar ne forma dee punto las observaciones }
  \FunctionTok{geom\_point}\NormalTok{(}\FunctionTok{aes}\NormalTok{(}\AttributeTok{colour =}\NormalTok{ Insumo), }
             \AttributeTok{size =} \DecValTok{2}\NormalTok{,}
             \AttributeTok{alpha =}\NormalTok{ .}\DecValTok{5}\NormalTok{,}
             \AttributeTok{show.legend =} \ConstantTok{FALSE}\NormalTok{, }
             \AttributeTok{position =} \FunctionTok{position\_jitter}\NormalTok{(}\AttributeTok{width =}\NormalTok{ altura,}
                                        \AttributeTok{height =} \DecValTok{0}\NormalTok{)) }\SpecialCharTok{+}
  \CommentTok{\# hacer los boxplot de las observaciones }
  \FunctionTok{geom\_boxplot}\NormalTok{(}\FunctionTok{aes}\NormalTok{(}\AttributeTok{group=}\NormalTok{Insumo, }\AttributeTok{color=}\NormalTok{Insumo),}
               \AttributeTok{width =}\NormalTok{ altura,}
               \AttributeTok{alpha =} \FloatTok{0.4}\NormalTok{, }
               \AttributeTok{show.legend =} \ConstantTok{FALSE}\NormalTok{, }
               \AttributeTok{outlier.shape =} \ConstantTok{NA}\NormalTok{,}
               \AttributeTok{position =} \FunctionTok{position\_nudge}\NormalTok{(}\AttributeTok{x =} \SpecialCharTok{{-}}\NormalTok{altura}\SpecialCharTok{*}\DecValTok{2}\NormalTok{)) }\SpecialCharTok{+}
   
   \CommentTok{\# poner nombre y limites en el eje y}
  \FunctionTok{scale\_y\_continuous}\NormalTok{(}\AttributeTok{name =} \StringTok{"Producción de bioetanol en m³"}\NormalTok{,}
                     \AttributeTok{breaks =} \FunctionTok{seq}\NormalTok{(}\DecValTok{0}\NormalTok{, }\DecValTok{80000}\NormalTok{, }\DecValTok{20000}\NormalTok{), }
                     \AttributeTok{limits =} \FunctionTok{c}\NormalTok{(}\DecValTok{0}\NormalTok{, }\DecValTok{80000}\NormalTok{)) }\SpecialCharTok{+}
   \CommentTok{\# Girar  el eje }
  \FunctionTok{coord\_flip}\NormalTok{() }\SpecialCharTok{+}
   
   \CommentTok{\# eliminar el nombre de  xlim del grafico }
   \FunctionTok{scale\_x\_discrete}\NormalTok{(}\AttributeTok{labels =} \ConstantTok{NULL}\NormalTok{, }\AttributeTok{breaks =} \ConstantTok{NULL}\NormalTok{) }\SpecialCharTok{+} 
   
   \CommentTok{\# colocar bonito el gráfico }
  \FunctionTok{labs}\NormalTok{(}\AttributeTok{x=}\StringTok{""}\NormalTok{,}\AttributeTok{title =} \StringTok{"Maíz vs Caña de azucar"}\NormalTok{, }
       \AttributeTok{subtitle =} \StringTok{"Producción de bioetanol por tipo de insumo"}\NormalTok{, }
       \AttributeTok{caption =} \StringTok{"fecha: 9 Sep 2022"}\NormalTok{ ) }\SpecialCharTok{+}
   \CommentTok{\# escoger forma del plano }
  \FunctionTok{theme\_minimal}\NormalTok{() }\SpecialCharTok{+}
   
  \FunctionTok{theme}\NormalTok{(}\AttributeTok{panel.grid.major.y =} \FunctionTok{element\_blank}\NormalTok{(),}
        \AttributeTok{legend.background =} \FunctionTok{element\_rect}\NormalTok{(}\AttributeTok{fill =} \StringTok{"white"}\NormalTok{, }\AttributeTok{color =} \StringTok{"white"}\NormalTok{))}
\end{Highlighting}
\end{Shaded}

\subsubsection{Output}\label{output}

\begin{center}\includegraphics{Trabajo-Práctico-I_files/figure-latex/unnamed-chunk-3-1} \end{center}

Se puede observar que hay mayor producción de bioetanol a base me maíz.

Ahora veamos un gráfico que divida la producción de bioetanol por años.

\subsection{Producción de bioetanol a base de materia prima por año
.}\label{producciuxf3n-de-bioetanol-a-base-de-materia-prima-por-auxf1o-.}

\subsubsection{Código}\label{cuxf3digo-1}

\begin{Shaded}
\begin{Highlighting}[]
\CommentTok{\# se realizó nos mismos pasos que en codigo anterior.}
\NormalTok{altura}\OtherTok{\textless{}{-}}\NormalTok{ .}\DecValTok{1}

\NormalTok{g2}\OtherTok{\textless{}{-}}\FunctionTok{ggplot}\NormalTok{(Bioetanol,}
       \FunctionTok{aes}\NormalTok{(}\AttributeTok{x =}\StringTok{""}\NormalTok{ ,}
           \AttributeTok{y =}\NormalTok{ Produccion,}
           \AttributeTok{fill =}\NormalTok{ Insumo)) }\SpecialCharTok{+}
  
\NormalTok{  introdataviz}\SpecialCharTok{::}\FunctionTok{geom\_flat\_violin}\NormalTok{(}\AttributeTok{trim=}\ConstantTok{FALSE}\NormalTok{, }
                                 \AttributeTok{alpha =} \FloatTok{0.4}\NormalTok{,}
                                 \AttributeTok{position =} \FunctionTok{position\_nudge}\NormalTok{(}\AttributeTok{x =}\NormalTok{ altura}\FloatTok{+.05}\NormalTok{)) }\SpecialCharTok{+}
  
  \FunctionTok{geom\_point}\NormalTok{(}\FunctionTok{aes}\NormalTok{(}\AttributeTok{colour =}\NormalTok{ Insumo),}
             \AttributeTok{size =} \DecValTok{2}\NormalTok{,}
             \AttributeTok{alpha =}\NormalTok{ .}\DecValTok{5}\NormalTok{,}
             \AttributeTok{show.legend =} \ConstantTok{FALSE}\NormalTok{, }
             \AttributeTok{position =} \FunctionTok{position\_jitter}\NormalTok{(}\AttributeTok{width =}\NormalTok{ altura, }\AttributeTok{height =} \DecValTok{0}\NormalTok{)) }\SpecialCharTok{+}
  
  \FunctionTok{geom\_boxplot}\NormalTok{(}\FunctionTok{aes}\NormalTok{(}\AttributeTok{group=}\NormalTok{Insumo, }\AttributeTok{color=}\NormalTok{Insumo),}
               \AttributeTok{width =}\NormalTok{ altura,}
               \AttributeTok{alpha =} \FloatTok{0.4}\NormalTok{, }
               \AttributeTok{show.legend =} \ConstantTok{FALSE}\NormalTok{, }
               \AttributeTok{outlier.shape =} \ConstantTok{NA}\NormalTok{,}
               \AttributeTok{position =} \FunctionTok{position\_nudge}\NormalTok{(}\AttributeTok{x =} \SpecialCharTok{{-}}\NormalTok{altura}\SpecialCharTok{*}\DecValTok{2}\NormalTok{)) }\SpecialCharTok{+}
  
  \FunctionTok{scale\_y\_continuous}\NormalTok{(}\AttributeTok{name =} \StringTok{"Producción de bioetanol en m³"}\NormalTok{,}
                     \AttributeTok{breaks =} \FunctionTok{seq}\NormalTok{(}\DecValTok{0}\NormalTok{, }\DecValTok{80000}\NormalTok{, }\DecValTok{20000}\NormalTok{), }
                     \AttributeTok{limits =} \FunctionTok{c}\NormalTok{(}\DecValTok{0}\NormalTok{, }\DecValTok{80000}\NormalTok{)) }\SpecialCharTok{+}
  \FunctionTok{coord\_flip}\NormalTok{() }\SpecialCharTok{+}
  \FunctionTok{facet\_wrap}\NormalTok{(}\SpecialCharTok{\textasciitilde{}}\FunctionTok{factor}\NormalTok{(Año)) }\SpecialCharTok{+}
  
  \FunctionTok{scale\_x\_discrete}\NormalTok{(}\AttributeTok{labels =} \ConstantTok{NULL}\NormalTok{, }\AttributeTok{breaks =} \ConstantTok{NULL}\NormalTok{) }\SpecialCharTok{+} 
  
  \FunctionTok{labs}\NormalTok{(}\AttributeTok{x=}\StringTok{""}\NormalTok{, }\AttributeTok{title =} \StringTok{"Maíz vs Caña de azucar"}\NormalTok{, }
       \AttributeTok{subtitle =} \StringTok{"Producción de bioetanol por tipo de insumo en los años 2017,2018 y 2019"}\NormalTok{, }
       \AttributeTok{caption =} \StringTok{"fecha: 9 Sep 2022"}\NormalTok{) }\SpecialCharTok{+}
  \FunctionTok{theme\_minimal}\NormalTok{() }\SpecialCharTok{+}
  
  \FunctionTok{theme}\NormalTok{(}\AttributeTok{panel.grid.major.y =} \FunctionTok{element\_blank}\NormalTok{(),}
       \CommentTok{\# legend.position = "bottom" ,}
        \AttributeTok{legend.background =} \FunctionTok{element\_rect}\NormalTok{(}\AttributeTok{fill =} \StringTok{"white"}\NormalTok{, }\AttributeTok{color =} \StringTok{"white"}\NormalTok{))}
\end{Highlighting}
\end{Shaded}

\subsubsection{Output}\label{output-1}

\begin{center}\includegraphics{Trabajo-Práctico-I_files/figure-latex/unnamed-chunk-5-1} \end{center}

Se puede observar en los gráficos de manera general, que la producción
de bioetanol a base de maíz es menos dispersa en relación a la
producción de bioetanol a base de caña de azucar.

En el segundo gráfico muestra que el producción de bioetanol a base de
maíz fue mayor en lo años 2018 y 2019, mientras que en el 2017 fue muy
similiar la producción de bioetanol entre los insumos.

\subsection{producción de bioetanol por insumo en los diferetes meses de
los
años.}\label{producciuxf3n-de-bioetanol-por-insumo-en-los-diferetes-meses-de-los-auxf1os.}

Ahora realizamos el gráfico para observar la producción de bioetanol por
insumo por mes de cada año, para esto utlizo la libreria de
\emph{ggplot2}.

\subsubsection{Código}\label{cuxf3digo-2}

\begin{Shaded}
\begin{Highlighting}[]
\FunctionTok{require}\NormalTok{(tidyverse)}

\CommentTok{\# convierto en factor el mes y escribo los meses ordinalmente para}
\CommentTok{\# que en el grafico aparezcan en orden.}

\NormalTok{Bioetanol}\SpecialCharTok{$}\NormalTok{Mes}\OtherTok{\textless{}{-}}\FunctionTok{factor}\NormalTok{(Bioetanol}\SpecialCharTok{$}\NormalTok{Mes, }
               \AttributeTok{levels =} \FunctionTok{c}\NormalTok{(}\StringTok{"Enero"}\NormalTok{,}\StringTok{"Febrero"}\NormalTok{,}\StringTok{"Marzo"}\NormalTok{,}\StringTok{"Abril"}\NormalTok{,}\StringTok{"Mayo"}\NormalTok{,}
                            \StringTok{"Junio"}\NormalTok{,}\StringTok{"Julio"}\NormalTok{,}\StringTok{"Agosto"}\NormalTok{,}\StringTok{"Septiembre"}\NormalTok{,}
                            \StringTok{"Octubre"}\NormalTok{,}\StringTok{"Noviembre"}\NormalTok{,}\StringTok{"Diciembre"}\NormalTok{))}

\CommentTok{\# convierto en factor la variable año}
\NormalTok{Bioetanol}\SpecialCharTok{$}\NormalTok{Año}\OtherTok{\textless{}{-}}\FunctionTok{as.factor}\NormalTok{(Bioetanol}\SpecialCharTok{$}\NormalTok{Año)}

\CommentTok{\# grafico la produccion de insumos por año}

\FunctionTok{library}\NormalTok{(ggplot2)}

\NormalTok{g3}\OtherTok{\textless{}{-}}\FunctionTok{ggplot}\NormalTok{(}\AttributeTok{data=}\NormalTok{Bioetanol, }
       \FunctionTok{aes}\NormalTok{(}\AttributeTok{x=}\FunctionTok{factor}\NormalTok{(Mes),}
           \AttributeTok{y=}\NormalTok{Produccion, }
           \AttributeTok{group=}\NormalTok{Insumo,}
           \AttributeTok{shape=}\NormalTok{Insumo,}
           \AttributeTok{color=}\NormalTok{Insumo,}
\NormalTok{           x)) }\SpecialCharTok{+} 
  \FunctionTok{geom\_line}\NormalTok{() }\SpecialCharTok{+} 
  \FunctionTok{geom\_point}\NormalTok{() }\SpecialCharTok{+}
  \FunctionTok{facet\_grid}\NormalTok{( .}\SpecialCharTok{\textasciitilde{}}\NormalTok{Año, }\AttributeTok{switch =} \StringTok{"x"}\NormalTok{) }\SpecialCharTok{+}  \CommentTok{\#scales="free", space="free"}
  \CommentTok{\#scale\_x\_discrete("Mes") +}
  \CommentTok{\#scale\_y\_continuous("Producción") +}
  \FunctionTok{theme}\NormalTok{(}\AttributeTok{axis.text.x =} \FunctionTok{element\_text}\NormalTok{(}\AttributeTok{angle =} \DecValTok{90}\NormalTok{))}\SpecialCharTok{+}
  \FunctionTok{labs}\NormalTok{(}\AttributeTok{x=}\StringTok{"Mes"}\NormalTok{, }\AttributeTok{y=}\StringTok{"Producción de bioetanol en m³"}\NormalTok{, }\AttributeTok{title =} \StringTok{"Maíz vs Caña de azucar"}\NormalTok{, }
       \AttributeTok{subtitle =} \StringTok{"Producción de bioetanol a base de maiz y caña de azucar"}\NormalTok{, }
       \AttributeTok{caption =} \StringTok{"fecha: 9 Sep 2022"}\NormalTok{) }
\end{Highlighting}
\end{Shaded}

\subsubsection{Output}\label{output-2}

\begin{center}\includegraphics{Trabajo-Práctico-I_files/figure-latex/unnamed-chunk-7-1} \end{center}

En el gráfico podemos ver que la producción de bioetanol a base de maíz
tiende a aumentar en los últimos meses de cada año, caso contrario con
la producción a base de caña de azucar. Además, se observa que la
producción a base de maíz, tiene menos variabilidad con respecto a la
pruducción a base de caña de azucar.

Para maíz, la máxima producción de bioetanol en el año 2017 se
desarrollo en los meses de diciembre y la mínima en el mes de mayo, para
el año 2018, la máxima se ubicó en el mes de diciembre y la mínima en el
mes de febrero,por último para el año 2019,abril fue el mes con mayor
producción de bioetanol con respecto a los 5 primeros meses y enero, el
mes con menor prodicción.

Para caña de azucar, el mes con mayor producción de bioetanol en el año
2017, se ubicó en el mes de octubre y la menor producción en el mes de
abril,para el año 2018, la maxima producción de bioetanol fue en el mes
de julio y la mínima en el mes de marzo, por último, mayo, fue el mes
con máxima producción de bioetanol para el 2019 y abril, el mes con
menor producción de bioetanol, cabe resaltar que en el 2019 solo se
tiene los datos de producción de bioetanol de los primeros 5 meses.

\subsection{Guardo figuras.}\label{guardo-figuras.}

\subsection{Referencias.}\label{referencias.}

Página para gráficos
({[}\url{https://r-graph-gallery.com/index.html}{]})

Análisis de
Temperatura({[}\url{file:///C:/Users/Usuario/AppData/Local/Temp/Rar$EXa0.115/datos_temperatura/analisis.html}{]})

\end{document}
